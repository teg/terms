\documentclass[11pt,a4paper]{article}

% A more pleasant font
\usepackage[T1]{fontenc} % use postscript type 1 fonts
\usepackage{textcomp} % use symbols in TS1 encoding
\usepackage[garamond]{mathdesign} % use a nice font

% Improves the text layout
\usepackage{microtype}

% To typeset derivations
\usepackage[lutzsyntax,noxy]{virginialake}

% Allow inclusion of graphics
\usepackage[pdftex]{graphicx}

% To get semantic brackets
\usepackage{stmaryrd}

% To use definitions, lemmas and theorems.
\usepackage{amsthm}
\theoremstyle{definition}
\newtheorem{definition}{Definition}
\theoremstyle{plain}
\newtheorem{lemma}[definition]{Lemma}
\newtheorem{theorem}[definition]{Theorem}
\theoremstyle{remark}
\newtheorem{remark}[definition]{Remark}

% Might not be needed, didn't check.
\usepackage{amsfonts}

\title{Atomic Flows for a Simply Typed Term Calculus}
\author{Tom Gundersen and Michel Parigot}

\begin{document}

\maketitle

\abstract{
A proof system for minimal logic with explicit, atomic structural rules is introduced, together with a normalisation procedure. Atomic flows are defined based on the proof system, and a flow rewriting system is given, which can simulate the proof normalisation. Furthermore, a term calculus is defined, together with a rewrite system corresponding to the proof normalisation. A denotation of the term calculus is given in terms of the lambda calculus, and it is shown that the new term calculus can simulate $\beta$-reduction and preserve strong normalisation. Lastly, the correspondence between the flow rewriting system and the term rewriting system is explored.
}

\section{Derivations}

Translate derivations from the standard system in natural deduction in seqeunt style with explicit structural rules, into open deduction. Add rules that make contraction and weakening atomic.

\subsection{Formulae}

\begin{definition}
	A  formula is defined to be
	\begin{itemize}
		\item the unit true: $\top$;
		\item an atom: $a$;
		\item a conjunction: $\vls(A.B)$; or
		\item an implication: $A\rightarrow B$,
	\end{itemize}
	where no conjunction or unit may occur to the left of an implication.
\end{definition}

We consider formulae modulo associativity and commutativity of conjunction, and we take $\top$ to be the unit for conjunction.

\subsection{Inference Rules}

\[
\vlinf{}{}{A\rightarrow\vlsbr(A.\Gamma)}{\Gamma}
\]

\[
\vlinf{}{}{B}{\vls(A.[A\rightarrow B])}
\]

\[
\vlinf{}{}{\top}{a}
\]

\[
\vlinf{}{}{\top}{A\rightarrow\top}
\]

\[
\vlinf{}{}{\vls(a.a)}{a}
\]

\[
\vlinf{}{}{\vls([A\rightarrow B].[A\rightarrow C])}{A\rightarrow\vlsbr(B.C)}
\]

\subsection{Composition of Derivations}

\[
\vls(\vlder{ }{ }{C}{A}\;\;.\;\;\vlder{ }{ }{D}{B})
\]

\[
A\;\;\rightarrow\;\;\vlder{ }{ }{C}{B}
\]

\[
\vlderivation
{
\vlde{ }{ }{C}
{
\vlde{ }{ }{B}
{
\vlhy{A}
}
}
}
\]

When we compose derivations vertically, we consider conjunctions to be associate, commutative, and have unit $\top$.

\subsubsection{Projections of Formulae}

\begin{definition}
	The first (resp., second) projection of a formula $\pi_1(A)$ (resp., $\pi_2(A)$), is defined as follows
	\begin{itemize}
		\item $\pi_1(\top)=\pi_2(\top)=\top$;
		\item $\pi_1(a)=\pi_2(a)=a$;
		\item $\pi_1(\vls(A.B))=A$ (resp., $\pi_2(\vls(A.B))=B$); and
		\item $\pi_1(A\rightarrow B)=\pi_1(A)\rightarrow\pi_1(B)$ (resp., $\pi_2(A\rightarrow B)=\pi_2(A)\rightarrow\pi_2(B)$).
	\end{itemize}
\end{definition}

\section{Flows}

Extract the flows from the derivation in the standard way, with the added restriciton that we distinguish between the left- and the right-hand side of the contraction.

\subsection{Reductions}

The reductions work in the standard way, with the one addition of two contractions meeting in an interaction, which are rewritten to two interactions joining up the two left-hand side edges and the two right-hand side edges, repsectivly.

It should be noted that this means that our reductions always preserve paths, where a valid path is defined by keeping a stack of left/right tokens, every time a contraction is entered from below a token is added and every time a contraction is entered from above a token is removed and the paths leaves by the edge indicated on the token.

This is very similar to somthing Ugo dal Lago told me about (the notion of path), so I need to look up that paper. My guess is that they use the paths to predict the size of the normal form of a term, we will get the same, but more explicitly.

\[
\vcenter{\hbox{\includegraphics{Figures/flow-beta-redex}}}
\quad\rightsquigarrow_\beta\quad
\vcenter{\hbox{\includegraphics{Figures/flow-beta}}}
\]

\[
\vcenter{\hbox{\includegraphics{Figures/flow-cont-cont-redex}}}
\quad\rightsquigarrow\quad
\vcenter{\hbox{\includegraphics{Figures/flow-cont-cont}}}
\]

\[
\vcenter{\hbox{\includegraphics{Figures/flow-cont-weak-redex}}}
\quad\rightsquigarrow\quad
\vcenter{\hbox{\includegraphics{Figures/flow-cont-weak}}}
\]

\[
\vcenter{\hbox{\includegraphics{Figures/flow-weak-weak-redex}}}
\quad\rightsquigarrow\quad
\]

\[
\vcenter{\hbox{\includegraphics{Figures/flow-weak-cont-redex}}}
\quad\rightsquigarrow\quad
\vcenter{\hbox{\includegraphics{Figures/flow-weak-cont}}}
\]

\section{Terms}

\newcommand{\FV}{{\mathop{\mathsf{FV}}}}
\newcommand{\Var}{{\mathop{\mathsf{Var}}}}
\newcommand{\terpair}[2]{{\langle{#1},{#2}\rangle}}
\newcommand{\terdel}[1]{{[\leftarrow{#1}]}}
\newcommand{\terdup}[3]{{[{#1},{#2}\leftarrow{#3}]}}

\begin{itemize}
	\item $x$;
	\item $\lambda x.u$, where $x\in\FV(u)$;
	\item $(u)t$, where $\Var(u)\cap\Var(t)=\emptyset$;
	\item $u\terdel{t}$, where $\Var(u)\cap\Var(t)=\emptyset$;
	\item $u\terdup{x}{y}{t}$, where $\Var(u)\cap\Var(t)=\emptyset, x,y\in\FV(u)$;
	\item $\epsilon$; and
	\item $\terpair{u}{v}$ where $\Var(u)\cap\Var(t)=\emptyset$.
\end{itemize}

\subsection{Projection of Terms}

\begin{definition}
	The first (resp., second) projection of a term, $t$, is denoted $\pi_1(t)$ (resp., $\pi_2(t)$), and is defined as follows
	\begin{itemize}
		\item $\pi_1(\epsilon)=\pi_2(\epsilon)=\epsilon$;
		\item $\pi_1(x)=\pi_2(x)=x$;
		\item $\pi_1(\terpair{u}{v})=u\terdel{v}$ (resp., $\pi_2(\terpair{u}{v})=v\terdel{u}$);
		\item $\pi_1(\lambda x.u)=\lambda x.\pi_1(u)$ (resp., $\pi_2(\lambda x.u)=\lambda x.\pi_2(u)$); and
		\item $\pi_1( (u)v)=(\pi_1(u))\pi_1(v)$ (resp., $\pi_2( (u)v)=(\pi_2(u))\pi_2(v)$).
	\end{itemize}
\end{definition}

\begin{lemma}
For every term $t$, $\FV(t)=\FV(\pi_1(t))=\FV(\pi_2(t))$.
\end{lemma}

\subsection{Denotation}

\newcommand{\terden}[1]{{\llbracket{#1}\rrbracket}}

\begin{definition}
	The denotation of a term $t$, is written $\terden{t}$, and is defined as follows
	\begin{itemize}
		\item $\terden{x}=x$;
		\item $\terden{\lambda x. u}=\lambda x.\terden{u}$;
		\item $\terden{(u)v}=(\terden{u})\terden{v}$;
		\item $\terden{u\terdel{v}}=\terden{u}$; and
		\item $\terden{u\terdup{x}{y}{v}}=\terden{u}\{x\leftarrow \terden{\pi_1(v)},y\leftarrow\terden{\pi_2(v)}\}$.
\end{itemize}
\end{definition}

Note that the denotation is not defined for all terms.

\begin{lemma}
	$\terden{u\{x\leftarrow t\}}=\terden{u}\{x\leftarrow\terden{t}\}$.
\end{lemma}

\subsection{Typing}

As before. For every typing derivation, there is a term.

\subsection{Reductions}

\subsubsection{$\beta$-reduction}

\[
(\lambda x.u)t \rightsquigarrow_\beta u\{x\leftarrow t\}
\]


\begin{lemma}
	If $u\rightsquigarrow_\beta v$, then $\terden{u}\rightsquigarrow_\beta\terden{v}$
\end{lemma}

\begin{proof}
	Assume, without loss of generality, that $u=(\lambda x.w)t$. We then have that $v=w\{x\leftarrow t\}$, and that
	\[
	\terden{(\lambda x.w)t}=(\lambda x.\terden{w})\terden{t}\rightsquigarrow_\beta\terden{w}\{x\leftarrow\terden{t}\}=\terden{w\{x\leftarrow\}}\;,
	\]
	as required.
\end{proof}

\subsubsection{Deletion}

\newcommand{\tempty}{{\mathtt{empty}}}

\begin{definition}
	The set of $\tempty$ terms is defined as
	\[\tempty:=\epsilon\;|\;\tempty\terdel{t}\;|\;\tempty\terdup{x}{y}{t}\;.\]
\end{definition}

\paragraph{Abstraction Reductions}

\begin{itemize}
	\item $u\terdel{\lambda x.t}\rightsquigarrow u\terdel{\lambda x.\epsilon\terdel{t}}$, if $t$ is not $\tempty$;
	\item $u\terdel{\lambda x.(u\terdel{t})}\rightsquigarrow u\terdel{(\lambda x.u)}\terdel{t}$, if $x\not\in\FV(t)$; and
	\item $u\terdel{(\lambda x.\epsilon\terdel{x})}\rightsquigarrow u$.
\end{itemize}

\paragraph{Substitution Reductions}

\begin{itemize}
	\item $u\terdel{t\terdel{v}}\rightsquigarrow u\terdel{t}\terdel{v}$;
	\item $u\terdel{t\terdup{x}{y}{v}}\rightsquigarrow u\terdel{t}\terdup{x}{y}{v}$; and
	\item $u\terdel{x}\terdup{x}{y}{t}\rightsquigarrow u\{y\leftarrow t\}$.
\end{itemize}

\paragraph{Pair and Empty Reductions}

\begin{itemize}
	\item $u\terdel{\epsilon}\rightsquigarrow u$; and
	\item $u\terdel{\terpair{t_1}{t_2}}\rightsquigarrow u\terdel{t_1}\terdel{t_2}$.
\end{itemize}

\begin{definition}
	The \emph{weight} of a deletion, $u\terdel{t}$, (resp, of a duplication, $u\terdup{x}{y}{t}$) in a term $v$ is defined to be the \emph{size} of $t$ in $v$, $|t|_v$:
	\begin{itemize}
		\item $|w|_v$, if $t$ is $w\terdel{w'}$ (resp., $w\terdup{x}{y}{w'}$);
		\item $1+|w|_v$, if $t$ is $\lambda x.w$;
		\item $|w|_v$, if $t$ is a variable bound by a duplication of $w$ in $v$;
		\item $0$, if $t$ is any otehr variable;
		\item $0$, if $t$ is an application;
		\item $1$, if $t$ is $\epsilon$; or
		\item $|w|_v+|w'|_v$ (resp., 1), if $t$ is $\terpair{w}{w'}$.
	\end{itemize}
\end{definition}

\begin{lemma}
The deletion reductions are strongly normalising.
\end{lemma}

\begin{proof}
Each deletion reduction rule replaces a deletion by zero or more of lower weight.
\end{proof}

\begin{lemma}
\[
u\terdel{t}\rightsquigarrow^\star u\terdel{(v_1)w_1}\cdots\terdel{(v_m)w_m}\terdel{x_1}\cdots\terdel{x_n}.
\]
\end{lemma}

\subsubsection{Duplication}

\newcommand{\tpair}{{\mathtt{pair}}}

\begin{definition}
	The set of $\tpair$ terms is defined as
	\[\tpair:=\terpair{u}{v}\;|\;\tpair\terdel{t}\;|\;\tpair\terdup{x}{y}{t}\;.\]
\end{definition}

\paragraph{Abstraction Reductions}

\begin{itemize}
	\item $u\terdup{z_1}{z_2}{\lambda y.t}\rightsquigarrow u\terdup{z_1}{z_2}{\lambda y.\terpair{x_1}{x_2}\terdup{x_1}{x_2}{t}}$, if $t$ is not a $\tpair$;
	\item $\lambda x.(u\terdup{y_1}{y_2}{y})\rightsquigarrow (\lambda x.u)\terdup{y_1}{y_2}{y}$, if $x\neq y$;
	\item $u\terdup{x_1}{x_2}{(\lambda y.\terpair{t_1}{t_2}\terdup{y_1}{y_2}{y})}\rightsquigarrow u\{x_1\leftarrow\lambda y_1.t_1,x_2\leftarrow\lambda y_2.t_2\}$, where $y_1\in\FV(t_1)$ and $y_2\in\FV(t_2)$; and
	\item $u\terdup{x_1}{x_2}{(\lambda y.\terpair{t_1}{t_2}\terdel{y})}\rightsquigarrow u\{x_1\leftarrow\lambda y.t_1\terdel{y},x_2\lambda y.t_2\terdel{y}\}$.
\end{itemize}

\paragraph{Substitution Reductions}

\begin{itemize}
	\item $u\terdup{x}{y}{t\terdel{v}}\rightsquigarrow u\terdup{x}{y}{t}\terdel{v}$; and
	\item $u\terdup{x_1}{x_2}{t\terdup{y_1}{y_2}{z}}\rightsquigarrow u\terdup{x_1}{x_2}{t}\terdup{y_1}{y_2}{z}$.
\end{itemize}

\paragraph{Pair and Empty Reductions}

\begin{itemize}
	\item $u\terdup{x}{y}{\epsilon}\rightsquigarrow\{x\rightarrow\epsilon,y\rightarrow\epsilon\}$; and
	\item $u\terdup{x}{y}{\terpair{t_1}{t_2}}\rightsquigarrow u\{x\leftarrow t_1,y\leftarrow t_2\}$.
\end{itemize}

\begin{lemma}
	For any two terms $u$ and $v$, such that $u\rightsquigarrow v$, we have that $\FV(u)=\FV(v)$.
\end{lemma}

\begin{lemma}
	The deletion and duplication reductions are strongly normalising.
\end{lemma}

\begin{lemma}
	$u\terdup{x}{y}{t}\rightsquigarrow^\star u\{x\rightarrow t',y\rightarrow t''\}\terdup{x_1}{y_1}{(v_1)w_1}\cdots\terdup{x_m}{y_m}{(v_m)w_m}\terdup{x_{m+1}}{y_{m+1}}{z_{m+1}}\cdots\terdup{x_n}{y_n}{z_n}=terdel{z_{n+1}}\cdots\terdel{z_p}$,
where $t'$ (resp., $t''$) is obtained from $\pi_1(t)$ (resp., $\pi_2(t)$) by replacing every free variable $z_{m+1},\dots,z_n$ with $x_{m+1},\dots,x_n$ (resp., $y_{m+1},\dots,y_n$) and every application $(v_1)w_1,\dots,(v_m)w_m$ with $x_1,\dots,x_m$ (resp., $y_1,\dots,y_m$).
\end{lemma}

\begin{lemma}
	If $u\rightsquigarrow v$, then $\terden{u}=\terden{v}$.
\end{lemma}

\begin{proof}
	The theorem follows trivially for the deletion reductions, as $\terden{u\terdel{v}}=\terden{u}$.

	The casees for the duplication reductions follows by a straight-forward case analysis.
\end{proof}

\begin{theorem}
	If\/ $\terden{u}\rightsquigarrow_\beta\terden{v}$, then $u'\rightsquigarrow_\beta v'$, where $\terden{u'}=\terden{u}$ and $\terden{v}=\terden{v'}$.
\end{theorem}

\begin{proof}
	Let $u'$ be the normal form of $u$ with respect to the deletion and duplication reductions. It is straight-forward to verify by a case-analysis that $u'$ contains the same $\beta$-redex as $\terden{u'}$, so $v'$ is the corresponding contractum.
\end{proof}


\section{Flows and Terms}

We associate every well-typed term with a flow in the obvious way, and show the correspondence between the two kinds of reductions.

\newcommand{\fl}{{\mathop{\mathsf{fl}}}}

\begin{definition}
	The flow of any well-typed term $t$, denoted $\fl(t)$, is the flow of the corresponding tying derivation.
\end{definition}

\begin{theorem}
	If $u\rightsquigarrow v$ (resp., $u\rightsquigarrow_\beta v$), then $\fl(u)\rightsquigarrow^\star\fl(v)$ (resp., $\fl(u)\rightsquigarrow^\star_\beta\fl(v)$).
\end{theorem}

\begin{theorem}
	For every term $u$ and flow $\alpha$, such that $\alpha$ is the normal form of $\fl(u)$, there exists a term $v$, such that $v$ is the normal form of $u$ and $\fl(v)=\alpha$.
\end{theorem}

We extend the correspondence to terms that can not be typed and see what happens.

\end{document}
