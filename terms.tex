\documentclass[11pt,a4paper]{article}

% A more pleasant font
\usepackage[T1]{fontenc} % use postscript type 1 fonts
\usepackage{textcomp} % use symbols in TS1 encoding
%\usepackage[garamond]{mathdesign} % use a nice font

% Improves the text layout
\usepackage{microtype}

% To typeset derivations
\usepackage[lutzsyntax,noxy]{virginialake}

% Allow inclusion of graphics
\usepackage[pdftex]{graphicx}

% To get semantic brackets
\usepackage{stmaryrd}

% To use definitions, lemmas and theorems.
\usepackage{amsthm}
\theoremstyle{definition}
\newtheorem{definition}{Definition}
\theoremstyle{plain}
\newtheorem{lemma}[definition]{Lemma}
\newtheorem{theorem}[definition]{Theorem}
\theoremstyle{remark}
\newtheorem{remark}[definition]{Remark}

% Might not be needed, didn't check.
\usepackage{amsfonts}

\title{An Atomic Term Calculus}
\author{Tom Gundersen, Willem Heijltjes and Michel Parigot}

\begin{document}

\maketitle

\abstract{
A proof system for minimal logic with explicit, atomic structural rules is introduced, together with a normalisation procedure. Furthermore, a term calculus is defined, together with an atomic rewrite system corresponding to the proof normalisation. A denotation of the term calculus is given in terms of the lambda calculus, and it is shown that the new term calculus simulates $\beta$-reduction and preserve strong normalisation.
}

\section{Terms}

\newcommand{\FV}{{\mathop{\mathsf{FV}}}}
\newcommand{\Var}{{\mathop{\mathsf{Var}}}}
\newcommand{\tertuple}[2]{{\langle{#1},\dots,{#2}{\rangle}}}
\newcommand{\tershare}[4]{{{#1}[{#2},\dots,{#3}\leftarrow{#4}]}}
\newcommand{\terpr}[2]{{\pi_{#1}({#2})}}

\begin{itemize}
	\item $x$;
	\item $\lambda x.u$, where $x\in\FV(u)$;
	\item $(u)t$, where $\Var(u)\cap\Var(t)=\emptyset$;
	\item $\tertuple{t_1}{t_n}$, where $\Var(t_i)\cap\Var(t_j)=\emptyset$ if $i\neq j$; and
	\item $\tershare{u}{x_1}{x_n}{t}$, where $x_1,\dots,x_n\in\FV(u)$ and $\Var(u)\cap\Var(t)=\emptyset$.
\end{itemize}

\subsection{Equivalence of Terms}

To be typeset. We want the monoid equations on sharing and we want commutativity of independent sharings.

\subsection{Projection of Terms}

\begin{definition}
	The i-th projection of a term, $t$, is denoted $\terpr{i}{t}$ and is defined as follows
	\begin{itemize}
		\item $\terpr{i}{x}=x$;
		\item $\terpr{i}{\lambda x.u}=\lambda x.\terpr{i}{u}$;
		\item $\terpr{i}{(u)v}=(\terpr{i}{u})\terpr{i}{v}$;
		\item $\terpr{i}{\tertuple{t_1}{t_n}}=t_i$ if $i\leq n$ and undefined otherwise; and
		\item $\terpr{i}{\tershare{u}{x_1}{x_n}{t}}$ is undefined.
	\end{itemize}
\end{definition}

%\begin{lemma}
%For every term $t$, $\FV(t)=\FV(\pi_1(t))=\FV(\pi_2(t))$.
%\end{lemma}

\subsection{Denotation}

\newcommand{\terden}[1]{{\llbracket{#1}\rrbracket}}
\newcommand{\tersubn}[5]{{{#1}\{{#2}\leftarrow{#3},\dots,{#4}\leftarrow{#5}\}}}

\begin{definition}
	The denotation of a term $t$, is written $\terden{t}$, and is defined as follows
	\begin{itemize}
		\item $\terden{x}=x$;
		\item $\terden{\lambda x. u}=\lambda x.\terden{u}$;
		\item $\terden{(u)v}=(\terden{u})\terden{v}$;
		\item $\terden{\tertuple{t_1}{t_n}}=\tertuple{\terden{t_1}}{\terden{t_n}}$; and
		\item $\terden{\tershare{u}{x_1}{x_n}{t}}=\terden{\tersubn{u}{x_1}{\terpr{1}{\terden{t}}}{x_n}{\terpr{i}{\terden{t}}}}$.
\end{itemize}
\end{definition}

We need (probably?) to restrict our definition of terms so that we can show that that every term has a denotation, but that should be simple enough (currently it is not true as we can build nonsensical terms where such as $t[x,y,z <- <u,v>]$).

\begin{lemma}
	$\terden{u\{x\leftarrow t\}}=\terden{u}\{x\leftarrow\terden{t}\}$.
\end{lemma}

The rewrite $\rightsquigarrow$ will be defined later, I'm just putting the lemmas early to give a hint at what we are aiming for.

\begin{lemma}
$\forall u,v$ such that $u\rightsquigarrow v$, $\terden{u}=\terden{v}$.
\end{lemma}

\subsection{Sharing Normal Form}

\begin{definition}
A term is \emph{shared} (this is misleading, find a better term!) if it is of the form $\tershare{u}{x_1}{x_n}{t}$.
\end{definition}

\begin{definition}
A term is on \emph{sharing normal form} if it is one of
\begin{itemize}
	\item $x$;
	\item $\lambda x.(\tershare{u}{z_1}{z_n}{x})$
	\item $(u)v$
	\item $\tertuple{t_1}{t_n}$; and
	\item $\tershare{\tershare{u}{x_{1,1}}{x_{1,k_1}}{x_1}\cdots}{x_{n,1}}{x_{n,k_n}}{x_n}$,
\end{itemize}
where $u,v,t_1,\dots,t_n$ are on sharing normal form and are not shared, and $\{x_{1,1},\dots,x_{1,k_1},\dots,x_{n,1},\dots,x_{n,k_n}\}\cap\{x_1,\dots,x_n\}=\emptyset.$
\end{definition}

In other words, a term is on sharing normal form if the only sharing that happens is at the outside of the term and sharing of bound variables immediately inside an abstraction.

\begin{lemma}
$\forall u, \exists v$ such that $u\rightsquigarrow^\star v$ and $v$ is on sharing normal form.
\end{lemma}

\subsection{$\beta$-reduction}

\[
(\lambda x.u)t \rightsquigarrow_\beta u\{x\leftarrow t\}
\]

\begin{lemma}
	If $u\rightsquigarrow_\beta v$, then $\terden{u}\rightsquigarrow_\beta\terden{v}$
\end{lemma}

%\begin{proof}
%	Assume, without loss of generality, that $u=(\lambda x.w)t$. We then have that $v=w\{x\leftarrow t\}$, and that
%	\[
%	\terden{(\lambda x.w)t}=(\lambda x.\terden{w})\terden{t}\rightsquigarrow_\beta\terden{w}\{x\leftarrow\terden{t}\}=\terden{w\{x\leftarrow t\}}\;,
%	\]
%	as required.
%\end{proof}

This one is a bit ugly, find a better way to express it:

\begin{lemma}
$\forall u,v$ such that $\terden{u}\rightsquigarrow_\beta\terden{v}$ and $u,v$ are on sharing normal form, then $\exists w$ such that $u\rightsquigarrow_\beta w$ and $w\rightsquigarrow^\star v$.
\end{lemma}

\subsection{Sharing Reductions}

\subsubsection{Abstraction}

\begin{itemize}
\item $\tershare{u}{x_1}{x_n}{\lambda y. t} \rightsquigarrow
      \tershare{u}{x_1}{x_n}{\lambda y.\tershare{\tertuple{z_1}{z_n}}{z_1}{z_n}{t}}$; and
\item $\tershare{u}{x_1}{x_n}{\lambda y.\tershare{\tertuple{t_1}{t_n}}{z_{1,1}}{z_{n,k_n}}{y}} \rightsquigarrow$
      \[\tershare{u}{x_1}{x_n}{\tertuple{\lambda z_1.\tershare{t_1}{z_{1,1}}{z_{1,1_k}}{z_1}}{\lambda z_n.\tershare{t_n}{z_{n,1}}{z_{n,n_k}}{z_n}}},\]
where $\{z_{i,1},\dots,z_{i,k_i}\}\subset\FV(t_i)$.
\end{itemize}

\subsubsection{Application}

\begin{itemize}
\item $\tershare{u}{x_1}{x_n}{(v)t} \rightsquigarrow \tershare{\tershare{\tershare{u}{x_1}{x_n}{\tertuple{(z_1)y_i}{(z_n)y_n}}}{z_1}{z_n}{v}}{y_1}{y_n}{t}$; and
\item $(\tershare{u}{x_1}{x_n}{v})t \rightsquigarrow \tershare{(u)t}{x_1}{x_n}{v}$.
\end{itemize}

\subsubsection{Tuple}

\begin{itemize}
\item $\tershare{u}{x_1}{x_n}{\tertuple{t_1}{t_n}} \rightsquigarrow u\{x_1\leftarrow t_1,\dots,x_n\leftarrow t_n\}$.
\end{itemize}

\subsubsection{Sharing}
\begin{itemize}
\item $\tershare{u}{x_1}{x_n}{\tershare{v}{y_1}{y_m}{t}} \rightsquigarrow \tershare{\tershare{u}{x_1}{x_n}{v}}{y_1}{y_n}{t}$.
\end{itemize}

\section{Derivations}

We might want to rethink this a bit, as the terms now use n-ary contractinos everywhere, as opposed to the binary and nullary ones used in the inference rules.

Translate derivations from the standard system in natural deduction in seqeunt style with explicit structural rules, into open deduction. Add rules that make contraction and weakening atomic.

\subsection{Formulae}

\begin{definition}
	A  formula is defined to be
	\begin{itemize}
		\item the unit true: $\top$;
		\item an atom: $a$;
		\item a conjunction: $\vls(A.B)$; or
		\item an implication: $A\rightarrow B$,
	\end{itemize}
	where no conjunction or unit may occur to the left of an implication.
\end{definition}

We consider formulae modulo associativity and commutativity of conjunction, and we take $\top$ to be the unit for conjunction.

\subsubsection{Projections of Formulae}

\begin{definition}
	The first (resp., second) projection of a formula $\pi_1(A)$ (resp., $\pi_2(A)$), is defined as follows
	\begin{itemize}
		\item $\pi_1(\top)=\pi_2(\top)=\top$;
		\item $\pi_1(a)=\pi_2(a)=a$;
		\item $\pi_1(\vls(A.B))=A$ (resp., $\pi_2(\vls(A.B))=B$); and
		\item $\pi_1(A\rightarrow B)=\pi_1(A)\rightarrow\pi_1(B)$ (resp., $\pi_2(A\rightarrow B)=\pi_2(A)\rightarrow\pi_2(B)$).
	\end{itemize}
\end{definition}

\subsection{Inference Rules}

\[
\vlinf{}{}{A\rightarrow\vlsbr(A.\Gamma)}{\Gamma}
\]

\[
\vlinf{}{}{B}{\vls(A.[A\rightarrow B])}
\]

\[
\vlinf{}{}{\top}{a}
\]

\[
\vlinf{}{}{\top}{A\rightarrow\top}
\]

\[
\vlinf{}{}{\vls(a.a)}{a}
\]

\[
\vlinf{}{}{\vls([A\rightarrow B].[A\rightarrow C])}{A\rightarrow\vlsbr(B.C)}
\]

\subsection{Composition of Derivations}

\[
\vls(\vlder{ }{ }{C}{A}\;\;.\;\;\vlder{ }{ }{D}{B})
\]

\[
A\;\;\rightarrow\;\;\vlder{ }{ }{C}{B}
\]

\[
\vlderivation
{
\vlde{ }{ }{C}
{
\vlde{ }{ }{B}
{
\vlhy{A}
}
}
}
\]

When we compose derivations vertically, we consider conjunctions to be associate, commutative, and have unit $\top$.

\end{document}

\section{Flows}

We might want to do this in a separate paper, where we have decomposed the application rules further, so that we'll get optimal reduction and a closer correspondence to the flow rewriting system.

Extract the flows from the derivation in the standard way, with the added restriciton that we distinguish between the left- and the right-hand side of the contraction.

\subsection{Reductions}

The reductions work in the standard way, with the one addition of two contractions meeting in an interaction, which are rewritten to two interactions joining up the two left-hand side edges and the two right-hand side edges, repsectivly.

It should be noted that this means that our reductions always preserve paths, where a valid path is defined by keeping a stack of left/right tokens, and every time a contraction is entered from below a token is added and every time a contraction is entered from above a token is removed and the paths leaves by the edge indicated on the token.

This is very similar to somthing Ugo dal Lago told me about (the notion of path), so I need to look up that paper. My guess is that they use the paths to predict the size of the normal form of a term, we will get the same, but more explicitly.

\[
\vcenter{\hbox{\includegraphics{Figures/flow-beta-redex}}}
\quad\rightsquigarrow_\beta\quad
\vcenter{\hbox{\includegraphics{Figures/flow-beta}}}
\]

\[
\vcenter{\hbox{\includegraphics{Figures/flow-cont-cont-redex}}}
\quad\rightsquigarrow\quad
\vcenter{\hbox{\includegraphics{Figures/flow-cont-cont}}}
\]

\[
\vcenter{\hbox{\includegraphics{Figures/flow-cont-weak-redex}}}
\quad\rightsquigarrow\quad
\vcenter{\hbox{\includegraphics{Figures/flow-cont-weak}}}
\]

\[
\vcenter{\hbox{\includegraphics{Figures/flow-weak-weak-redex}}}
\quad\rightsquigarrow\quad
\]

\[
\vcenter{\hbox{\includegraphics{Figures/flow-weak-cont-redex}}}
\quad\rightsquigarrow\quad
\vcenter{\hbox{\includegraphics{Figures/flow-weak-cont}}}
\]

\section{Flows and Terms}

We associate every well-typed term with a flow in the obvious way, and show the correspondence between the two kinds of reductions.

\newcommand{\fl}{{\mathop{\mathsf{fl}}}}

\begin{definition}
	The flow of any well-typed term $t$, denoted $\fl(t)$, is the flow of the corresponding typing derivation.
\end{definition}

\begin{theorem}
	If $u\rightsquigarrow v$ (resp., $u\rightsquigarrow_\beta v$), then $\fl(u)\rightsquigarrow^\star\fl(v)$ (resp., $\fl(u)\rightsquigarrow^\star_\beta\fl(v)$).
\end{theorem}

\begin{theorem}
	For every term $u$ and flow $\alpha$, such that $\alpha$ is the normal form of $\fl(u)$, there exists a term $v$, such that $v$ is the normal form of $u$ and $\fl(v)=\alpha$.
\end{theorem}

We extend the correspondence to terms that can not be typed and see what happens.
