\documentclass[11pt,a4paper]{article}

% A more pleasant font
\usepackage[T1]{fontenc} % use postscript type 1 fonts
\usepackage{textcomp} % use symbols in TS1 encoding
\usepackage[garamond]{mathdesign} % use a nice font

% Improves the text layout
\usepackage{microtype}

% To typeset derivations
\usepackage[lutzsyntax,noxy]{virginialake}

% Allow inclusion of graphics
\usepackage[pdftex]{graphicx}

% To get semantic brackets
\usepackage{stmaryrd}

% To use definitions, lemmas and theorems.
\usepackage{amsthm}
\theoremstyle{definition}
\newtheorem{definition}{Definition}
\theoremstyle{plain}
\newtheorem{lemma}[definition]{Lemma}
\newtheorem{theorem}[definition]{Theorem}
\theoremstyle{remark}
\newtheorem{remark}[definition]{Remark}

% Might not be needed, didn't check.
\usepackage{amsfonts}

\title{Atomic Flows for a Simply Typed Term Calculus}
\author{Tom Gundersen and Michel Parigot}

\begin{document}

\maketitle

\abstract{
A proof system for minimal logic with explicit, atomic structural rules is introduced, together with a normalisation procedure. Atomic flows are defined based on the proof system, and a flow rewriting system is given, which can simulate the proof normalisation. Furthermore, a term calculus is defined, together with a rewrite system corresponding to the proof normalisation. A denotation of the term calculus is given in terms of the lambda calculus, and it is shown that the new term calculus can simulate $\beta$-reduction and preserve strong normalisation. Lastly, the correspondence between the flow rewriting system and the term rewriting system is explored.
}

\section{Derivations}

Translate derivations from the standard system in natural deduction in seqeunt style with explicit structural rules, into open deduction. Add rules that make contraction and weakening atomic.

\subsection{Formulae}

\begin{definition}
	A  formula is defined to be
	\begin{itemize}
		\item the unit true: $\top$;
		\item an atom: $a$;
		\item a conjunction: $\vls(A.B)$; or
		\item an implication: $A\rightarrow B$,
	\end{itemize}
	where no conjunction or unit may occur to the left of an implication.
\end{definition}

We consider formulae modulo associativity and commutativity of conjunction, and we take $\top$ to be the unit for conjunction.

\subsubsection{Projections of Formulae}

\begin{definition}
	The first (resp., second) projection of a formula $\pi_1(A)$ (resp., $\pi_2(A)$), is defined as follows
	\begin{itemize}
		\item $\pi_1(\top)=\pi_2(\top)=\top$;
		\item $\pi_1(a)=\pi_2(a)=a$;
		\item $\pi_1(\vls(A.B))=A$ (resp., $\pi_2(\vls(A.B))=B$); and
		\item $\pi_1(A\rightarrow B)=\pi_1(A)\rightarrow\pi_1(B)$ (resp., $\pi_2(A\rightarrow B)=\pi_2(A)\rightarrow\pi_2(B)$).
	\end{itemize}
\end{definition}

\subsection{Inference Rules}

\[
\vlinf{}{}{A\rightarrow\vlsbr(A.\Gamma)}{\Gamma}
\]

\[
\vlinf{}{}{B}{\vls(A.[A\rightarrow B])}
\]

\[
\vlinf{}{}{\top}{a}
\]

\[
\vlinf{}{}{\top}{A\rightarrow\top}
\]

\[
\vlinf{}{}{\vls(a.a)}{a}
\]

\[
\vlinf{}{}{\vls([A\rightarrow B].[A\rightarrow C])}{A\rightarrow\vlsbr(B.C)}
\]

\subsection{Composition of Derivations}

\[
\vls(\vlder{ }{ }{C}{A}\;\;.\;\;\vlder{ }{ }{D}{B})
\]

\[
A\;\;\rightarrow\;\;\vlder{ }{ }{C}{B}
\]

\[
\vlderivation
{
\vlde{ }{ }{C}
{
\vlde{ }{ }{B}
{
\vlhy{A}
}
}
}
\]

When we compose derivations vertically, we consider conjunctions to be associate, commutative, and have unit $\top$.

\section{Flows}

Extract the flows from the derivation in the standard way, with the added restriciton that we distinguish between the left- and the right-hand side of the contraction.

\subsection{Reductions}

The reductions work in the standard way, with the one addition of two contractions meeting in an interaction, which are rewritten to two interactions joining up the two left-hand side edges and the two right-hand side edges, repsectivly.

It should be noted that this means that our reductions always preserve paths, where a valid path is defined by keeping a stack of left/right tokens, every time a contraction is entered from below a token is added and every time a contraction is entered from above a token is removed and the paths leaves by the edge indicated on the token.

This is very similar to somthing Ugo dal Lago told me about (the notion of path), so I need to look up that paper. My guess is that they use the paths to predict the size of the normal form of a term, we will get the same, but more explicitly.

\[
\vcenter{\hbox{\includegraphics{Figures/flow-beta-redex}}}
\quad\rightsquigarrow_\beta\quad
\vcenter{\hbox{\includegraphics{Figures/flow-beta}}}
\]

\[
\vcenter{\hbox{\includegraphics{Figures/flow-cont-cont-redex}}}
\quad\rightsquigarrow\quad
\vcenter{\hbox{\includegraphics{Figures/flow-cont-cont}}}
\]

\[
\vcenter{\hbox{\includegraphics{Figures/flow-cont-weak-redex}}}
\quad\rightsquigarrow\quad
\vcenter{\hbox{\includegraphics{Figures/flow-cont-weak}}}
\]

\[
\vcenter{\hbox{\includegraphics{Figures/flow-weak-weak-redex}}}
\quad\rightsquigarrow\quad
\]

\[
\vcenter{\hbox{\includegraphics{Figures/flow-weak-cont-redex}}}
\quad\rightsquigarrow\quad
\vcenter{\hbox{\includegraphics{Figures/flow-weak-cont}}}
\]

\section{Terms}

\newcommand{\FV}{{\mathop{\mathsf{FV}}}}
\newcommand{\Var}{{\mathop{\mathsf{Var}}}}

\begin{itemize}
	\item $x$;
	\item $\lambda x.u$, where $x\in\FV(u)$;
	\item $(u)t$, where $\Var(u)\cap\Var(t)=\emptyset$;
	\item $u[\leftarrow t]$, where $\Var(u)\cap\Var(t)=\emptyset$;
	\item $u[x,y\leftarrow t]$, where $\Var(u)\cap\Var(t)=\emptyset, x,y\in\FV(u)$;
	\item $\epsilon$; and
	\item $\langle u,v\rangle$ where $\Var(u)\cap\Var(t)=\emptyset$.
\end{itemize}

\subsection{Projections}

As before.

\subsection{Denotation}

As before.

\subsection{Typing}

As before. For every typing derivation, there is a term.

\subsection{Reductions}

\subsubsection{$\beta$-reduction}

\[
(\lambda x.u)t \rightsquigarrow_\beta u\{x\leftarrow t\}
\]


\begin{lemma}
	If $u\rightsquigarrow_\beta v$, then $\llbracket u\rrbracket \rightsquigarrow_\beta\llbracket v\rrbracket$
\end{lemma}

\subsubsection{Deletion}

\newcommand{\tempty}{{\mathtt{empty}}}

\begin{definition}
	The set of $\tempty$ terms is defined as
	\[\tempty:=\epsilon\;|\;\tempty[\leftarrow t]\;|\;\tempty[x,y\leftarrow t]\;.\]
\end{definition}

\paragraph{Abstraction Reductions}

\begin{itemize}
	\item $u[\leftarrow\lambda x.t]\rightsquigarrow u[\leftarrow\lambda x.\epsilon[\leftarrow t]]$, if $t$ is not $\tempty$;
	\item $\lambda x.(u[\leftarrow y])\rightsquigarrow (\lambda x.u)[\leftarrow y]$, if $x\neq y$; and
	\item $u[\leftarrow(\lambda x.\epsilon[\leftarrow x])]\rightsquigarrow u$.
\end{itemize}

\paragraph{Substitution Reductions}

\begin{itemize}
	\item $u[\leftarrow t[\leftarrow v]]\rightsquigarrow u[\leftarrow t][\leftarrow v]$;
	\item $u[\leftarrow t[x,y\leftarrow v]]\rightsquigarrow u[\leftarrow t][x,y\leftarrow v]$; and
	\item $u[\leftarrow x][x,y\leftarrow t]\rightsquigarrow u\{y\leftarrow t\}$.
\end{itemize}

\paragraph{Pair and Empty Reductions}

\begin{itemize}
	\item $u[\leftarrow\epsilon]\rightsquigarrow u$; and
	\item $u[\leftarrow\langle t_1,t_2\rangle]\rightsquigarrow u[\leftarrow t_1][\leftarrow t_2]$.
\end{itemize}

Strongly normalising.

\subsubsection{Duplication}

\newcommand{\tpair}{{\mathtt{pair}}}

\begin{definition}
	The set of $\tpair$ terms is defined as
	\[\tpair:=\langle u,v\rangle\;|\;\tpair[\leftarrow t]\;|\;\tpair[x,y\leftarrow t]\;.\]
\end{definition}

\paragraph{Abstraction Reductions}

\begin{itemize}
	\item $u[z_1,z_2\leftarrow\lambda y.t]\rightsquigarrow u[z_1,z_2\leftarrow\lambda y.\langle x_1,x_2\rangle[x_1,x_2\leftarrow t]]$, if $t$ is not a $\tpair$;
	\item $\lambda x.(u[y_1,y_2\leftarrow y])\rightsquigarrow (\lambda x.u)[y_1,y_2\leftarrow y]$, if $x\neq y$;
	\item $u[x_1,x_2\leftarrow(\lambda y.\langle t_1,t_2\rangle[y_1,y_2\leftarrow y])]\rightsquigarrow u\{x_1\leftarrow\lambda y_1.t_1,x_2\leftarrow\lambda y_2.t_2\}$, where $y_1\in\FV(t_1)$ and $y_2\in\FV(t_2)$; and
	\item $u[x_1,x_2\leftarrow(\lambda y.\langle t_1,t_2\rangle[\leftarrow y])]\rightsquigarrow u\{x_1\leftarrow\lambda y.t_1[\leftarrow y],x_2\lambda y.t_2[\leftarrow y]\}$.
\end{itemize}

\paragraph{Substitution Reductions}

\begin{itemize}
	\item $u[x,y\leftarrow t[\leftarrow v]]\rightsquigarrow u[x,y\leftarrow t][\leftarrow v]$; and
	\item $u[x_1,x_2\leftarrow t[y_1,y_2\leftarrow z]]\rightsquigarrow u[x_1,x_2\leftarrow t][y_1,y_2\leftarrow z]$.
\end{itemize}

\paragraph{Pair and Empty Reductions}

\begin{itemize}
	\item $u[x,y\leftarrow\epsilon]\rightsquigarrow\{x\rightarrow\epsilon,y\rightarrow\epsilon\}$; and
	\item $u[x,y\leftarrow\langle t_1,t_2\rangle]\rightsquigarrow u\{x\leftarrow t_1,y\leftarrow t_2\}$.
\end{itemize}

Strongly normalising.

\begin{lemma}
	If $u$ is on normal form with respect to deletion and duplication, and $\llbracket u\rrbracket\rightsquigarrow_\beta v'$, then there is a $v$, such that $u\rightsquigarrow_\beta v$ and $\llbracket v\rrbracket=v'$.
\end{lemma}


\section{Flows and Terms}

We associate every well-typed term with a flow in the obvious way, and show the correspondence between the two kinds of reductions.

We extend the correspondence to terms that can not be typed and see what happens.

\end{document}
